\paragraph{Example: }What are the basic properties of an $f/10$, $10$"
telescope?
\begin{itemize}
	\item Focal length:
	$$
		F/\# = \frac{f}{D} \; \rightarrow \; 10 \times 10 \times 2.54 = 254
		\mathrm{cm}
	$$
	\item Plate scale:
	$$
		\frac{\mathrm{d}\theta}{\mathrm{d}s} = \frac{1}{f} = \frac{1}{254} = 
		0.22 \, ^\circ / \mathrm{cm}
	$$
	\item Light gathering power compared to eye:
	$$
		\mathrm{LGP} = \frac{25.4^2}{0.12^2} = 4.4\times 10^4
	$$
	\item Resolving power:
	$$
		\theta = 1.22\frac{\lambda}{D} = 0.5 \mathrm{"}
	$$
\end{itemize}

\section{The Atmosphere and Detectors}
The atmosphere is a right pain. It causes:
\begin{itemize}
	\item Absorption
	\item Refraction and dispersion
	\item Emission \begin{itemize}
		\item Thermal
		\item Rayleigh scattering
		\item Flourescent emission
	\end{itemize}
	\item Turbulance (twinkle)
\end{itemize}

\subsection{Correcting for atmospheric absorption}
We have that:
$$
	m_{corr} = m_{obs} - A_\lambda \sec z
$$
Where $A_\lambda$ is the number of air volumes you're looking through, and $z$
is the angle with respect to straight up.

\subsection{Atmospheric refraction}
This is what causes the `green flash'. The angle that the light is refracted by,
$r$, is given by:
$$
	r = (n-1)\tan(z)
$$
Where $z$ is the same as above. This means that any object that is not directly
above us shows a spectrum ($n$ varies by colour).

\subsection{Atmospheric emission}
We see very bright emission lines from OH in the upper atmosphere, as they
transfrom between rotational and vibrational energy levels. These are mainly
in the infa-red, however we really want to look in the IR - so we just ignore it
and build telescopes anyway. Most of the IR comes from the heat of the
telescopes themselves and as such we need to keep them very cool. We want to
observe in the IR because it allows us to see through the dust in the center of
our galaxy.

